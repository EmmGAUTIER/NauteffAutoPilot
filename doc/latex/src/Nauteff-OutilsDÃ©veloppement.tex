\documentclass[a4paper,11pt]{report}
\usepackage[utf8]{inputenc}
\usepackage[T1]{fontenc}
\usepackage[utf8]{inputenc}
\usepackage{lmodern}
%%\usepackage[francais]{babel}
\usepackage[french]{babel}
\usepackage{numprint}
\usepackage{glossaries}
\title{Nauteff P-1 Spécification}
\author{Emmanuel Gautier}
\makeglossaries

\begin{document}
\maketitle

\begin{abstract}
Ce document contient les informations permettant la génération du code du prototype Nauteff P-1.
\end{abstract}

\tableofcontents
\chapter{Buts du document}
Ce document contient les informations permettant d'installer, de configurer et utiliser les outils de développement. Ces outils sont la chaîne de compilation croisée GCC, eclipse et openocd. 
\chapter{Documents applicables et de référence}
\chapter{Terminologie}
\newglossaryentry{NMEA}
{
	name={NMEA}, % le terme à référencer (l'entrée qui apparaitra dans le glossaire)
	description={National Marine Electronics Association}, % la description du terme(sans retour à la ligne)
	sort={NMEA}, % si le mot contient des caractère spéciaux, ils ne seront pas pris en compte
	plural={NMEA} % la forme plurielle du terme
}
\newglossaryentry{I2C}{name=I2C, description={Inter-Integrated Circuit en englais. C'est un bus de communication série à 2 fils à courte distance}}

\printglossary

\chapter{Environnement de développement}
Le développement est réalisé avec des PC fonctionnant sous Linux/debian.
Au bureau il est réalisé avec les matériels suivant :
\begin{itemize}
	\item un PC permettant de faire fonctionner eclipse et la chaîne de compilation;
	\item un oscilloscope pour certaines mise au point ;
	\item occasionnellement un voltmètre;
	\item petit outillage: pinces tournevis, pince à dénuder,\ldots;
\end{itemize}

Lors des essais en mer du prototype un petit PC consommant peu
mais ne permettant pas d'utiliser eclipse permet de faire des
modifications du logiciel et, en particulier, des ajustements
de paramètres.Il embarque la chaîne de compilation et openocd.
La compilation est réalisée avec la commande make.

\begin{tabular}{|rcl|}
	\hline 
	&  &  \\ 
	\hline 
	OS & : & Debian Burster / amd64 \\ 
	eclipse& : & 2021 mars \\ 
	& : &  \\ 
	& : &  \\ 
	& : &  \\ 
	\hline 
\end{tabular} 

\chapter{Outils de développement}
\section{Compilateur et éditeur de liens arm-none-eabi-gcc}
\subsection{Installation}
L'installation est faite avec apt-get ou synaptic ou autre. Il faut les paquets suivants: 

\begin{tabular}{|c|p{6cm}|}
	\hline 
    binutils-arm-none-eabi & assembleur, éditeur de liens et outils binaires GNU pour les processeurs ARM Cortex-A/R/M\\
    \hline
    gcc-arm-none-eabi & GCC cross compiler for ARM Cortex-A/R/M processors\\
    \hline
    gdb-arm-none-eabi & GNU debugger for ARM Cortex-A/R/M processors\\
    \hline
    libstdc++-arm-none-eabi-newlib & GNU Standard C++ Library v3 for ARM Cortex-A/R/M processors (newlib)\\
    \hline
    libnewlib-arm-none-eabi & C library and math library compiled for bare metal using Cortex A/R/M\\
	\hline 
\end{tabular} 

\section{EDI eclipse}
l'EDI est Eclipse C++ téléchargeable sur le site eclipse.org.
\paragraph{}
Choisir "Eclipse IDE for Embedded C/C++ developers".

Plugins : TexLipse

Ajouter le plugin     C/C++ GDB Hardware Debugging Developer Resources.
Ce plugin est accessible a http://download.eclipse.org/tools/cdt/releases/NN.MM.
il faut utiliser le filtre gdb

\paragraph{}
Dans Eclipse marketplace demandes openocd et choisir GNU MCU Eclipse

\section{openocd}
Le logiciel qui permet d'accéder à la sonde est openocd version 0.9.0.

\chapter{Génération du projet}

<File>, <New project>, <C/C++ Project>, <C managed build>, <Debug> \& <Release>,

Project Name: NauteffAutopilot, <ARM Cross GCC>\\
\begin{tabular}{|l|l|}
	\hline 
	Chip Family &  STM30x/31x\\ 
	Flash size & 64k \\ 
	RAM size &  12k \\ 
	CCM size &  4k \\ 
	External clock&   8000000\\ 
	Content & Empty \\ 
	Use system calls & Freestanding (no Posix system calls) \\ 
	Trace output & ARM ITM (SWIO) pour essayer \\ 
	Check some \& most warnings & Oui \\ 
	Enable -Werror & Non \\ 
	Use -Og on debug & Oui \\ 
	Use newlib nano &  Oui\\ 
	Exclude unused & Oui \\ 
	Use link optimizations & Non \\ 
	Include folder & inc\\
	Source folder & src\\
	System folder & system\\
	CMSIS Library folder & cmsis \\
	C library folder & newlib\\
	linker script folder & ldscripts\\
	Select configuration & Debug \& release \\
	\hline 
\end{tabular} 
Dans <C/C++> <Build> <Tools settings>
\begin{tabular}{|l|l|}
	\hline 
	ARM family & cortex-m4\\ 
	Instruction set & Thumb\\ 
	Float ABI & FP instructions (hard)\\ 
	& \\ 
	& \\ 
	& \\ 
	& \\ 
	& \\ 
	& \\ 
	\hline 
\end{tabular} 




Créer les répertoires src et inc avec <File> <New Folder>

Dans la vue Project explorer sélectionner le projet puis <Project>, <Properties>
<C/C++ Build>, <Settings>
<Cross Settings> Prefix : arm-none-eabi- Path /usr/bin
<Cross GCC Compiler> All options : "-O0 -g3 -Wall -c -fmessage-length=0"


Commande de compilation :
arm-none-eabi-gcc -g3 -Wall -fno-math-errno -mcpu=cortex-m4 -mthumb -mfloat-abi=hard -fsingle-precision-constant -mfpu=fpv4-sp-d16 -specs=nosys.specs -specs=nano.specs -Iinclude -lm -lg -lc -Tnauteff.ld -Xlinker -Map -Xlinker nauteff.map



\end{document}
