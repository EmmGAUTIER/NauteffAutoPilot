\documentclass[a4paper,11pt]{report}
\usepackage[utf8]{inputenc}
\usepackage[T1]{fontenc}
\usepackage[utf8]{inputenc}
\usepackage{lmodern}
%%\usepackage[francais]{babel}
\usepackage[french]{babel}
\usepackage{numprint}
\usepackage{glossaries}
\title{Nauteff P-1 Spécification}
\author{Emmanuel Gautier}


%%\newglossaryentry{MEMS}{name=Nom, description={}}

\newglossaryentry{NMEA}
{
    name={NMEA}, % le terme à référencer (l'entrée qui apparaitra dans le glossaire)
    description={National Marine Electronics Association}, % la description du terme(sans retour à la ligne)
    sort={NMEA}, % si le mot contient des caractère spéciaux, ils ne seront pas pris en compte
    plural={NMEA} % la forme plurielle du terme
    }

\newglossaryentry{I2C}{name=I2C, description={Inter-Integrated Circuit en englais. C'est un bus de communication série à 2 fils à courte distance}}

\newglossaryentry{MEMS} {
    name=MEMS,
    description={Micro Electro Mechanicals Systems}
    }

\newglossaryentry{UART} {
    name=UART,
    description={Universal Asynchronous Receiver Transmitter soit émetteur-récepteur asynchrone universel, aussi appelée liaison série.}
    }

\newglossaryentry{pilonnement} {
    name=Pilonnement,
    sort={PILONNEMENT},
    description={Mouvement vertical du navire, c.à.d. de bas en haut et de haut en bas}
    }

\newglossaryentry{cavalement} {
    name=Cavalement,
    sort={CAVALEMENT},
    description={Mouvement le long de l'axe longitudinal du navire, c.à.d. d'avant en arrière, ce mouvement correspond à une accélération ou un ralentissement}
    }

\newglossaryentry{embardee} {
    name=embardée,
    sort={EMBARDEE},
    description={Mouvement latéral du navire, c.à.d. vers babord ou tribord, souvent liée à la dérive}
    }

\newglossaryentry{babord} {
    name=bâbord,
    sort={BABORD},
    description={désigne le côté gauche d'un navire, en tournant le dos à la poupe}
    }

\newglossaryentry{tribord} {
	name=tribord,
	description={le côté droit d'un navire, en tournant le dos à la poupe}
    }

\newglossaryentry{lacet} {
	name=lacet,
	description={mouvement de rotation par rapport à l'axe vertical du navire}
    }

\newglossaryentry{roulis} {
	name=roulis,
	description={ Mouvement d'oscillation d'un bateau autour de l'axe longitudinal (en général sous l'effet de la houle ou du vent)}
    }

\newglossaryentry{tangage} {
	name=Tangage,
	description={mouvement de rotation par rapport à l'axe transversal du navire}
    }

\newglossaryentry{gite} {
	name=gîte,
	description={inclinaison latérale d'un navire}
    }


\makeglossaries
\makeglossary

\begin{document}
\maketitle

\begin{abstract}
Ce document contient des principes conduisant à la spécification du prototype Nauteff P-1.
Il est une première étape à la rédaction de la spécification du Nauteff.
\end{abstract}

\tableofcontents
\printglossaries

\chapter{Buts du document}
\section{Buts}
Ce document contient les réflexions préliminaires à la spécification 
des principes de sécurité et des algorithmes de calculs.
du Nauteff. Il sert à la rédaction de sa spécification.

\section{Guide de lecture}


%\chapter{Documents applicables et de référence}
%\chapter{Terminologie}

\chapter{Principes de sécurité}
\section {Perte totale d'alimentation électrique}
La conséquence est la perte de contrôle du navire par le pilote.
En cas de perte totale d'alimentation le Nauteff doit émettre une alarme
sonore avec l'énergie électrique restante. Faut-il bloquer la barre ?
Évènement grave nécessitant un action urgent du navigateur.

\section {Tension de l'alimentation électrique faible}
Cet évènement est généralement du à une décharge des batteries,
il peut aussi être du à des mauvaises connexions ou des fils de section insuffisante.
La conséquence immédiate une est consommation accrue de courant et une moindre efficacité
du pilote surtout si la barre est dure, à plus long terme les conséquences peuvent être :
une décharge des batteries, la perte de contrôle du navire par le pilote, des désordres
sur le circuit électrique.
Le Nauteff doit émettre une alarme pour alerter le navigateur. 

\section {Tension de l'alimentation électrique trop forte}
La conséquence peut être une destruction du pilote.

Le Nauteff doit émettre une alarme pour alerter le navigateur.
Le Nauteff supporte une surtension importante

\section {Courant du moteur très élevé}
Le courant dépasse celui du moteur bloqué.
Ce cas est souvent du à un court-circuit.

Probabilité : moyenne lors des branchements, faible ensuite.

Conséquences : Perte de contrôle du navire par le pilote, perturbations sur le circuit électrique, dommages sévères au circuit électrique et risque d'incendie.

Actions: Arrêt immédiat de la commande du moteur, émission d'alarme.

\section {Courant du moteur très faible ou nul}
Ce cas est souvent du à un débranchement de l'actionneur.

Probabilité : moyenne.

Conséquences : Perte de contrôle du navire par le pilote, perturbations sur le circuit électrique, dommages sévères au circuit électrique et risque d'incendie.

Actions: Arrêt immédiat de la commande du moteur, émission d'alarme.

\section {Perte d'informations d'instruments de navigation}

probabilité:moyenne.

Conséquence en mode veille ou suivi de cap : performance du pilote réduite

Conséquence en mode suivi de vent ou de route par GPS : Perte de la fonction de suivi de vent ou de route.

Action sans suivi  vent ou route GPS : information visuelle sans alarme.

Action avec suivi de vent ou route GPS : passage en mode suivi de Cap avec la cap courant,
émission d'une alarme, reprise du mode précédant dès le retour de l'information et extinction de l'alarme.


\subsection{Écart important entre le cap ou la route et un consigne}
Cet évènement survient en mode automatique, de nombreuses causes peuvent le provoquer.

probabilité: moyenne

conséquences: perte de contrôle du navire par le pilote

action : alarme

\subsection{Fortes variations de cap en mode automatique}

conséquences: fatigue de l'équipage, du bateau, mauvaise tenue du cap ou de la route, consommation électrique du pilote plus élevée, 

Action: Augmentation du gain du pilote s'il est trop faible, affichage d'informations, alarme.
Dans le cas d'un voilier cherchant à gagner au vent, il faut parfois le laisser lofer
lors des surventes pour limiter la gîte et si possible avoir un meilleur gain au vent.
Des mouvements importants de la barre augmentent la consommation du pilote
et des angles de barre importants freinent le bateau et risquent de faire décrocher
le safran. Il faudra prévoir des paramètres pour l'algorithme du pilote.

\subsection{Actionneur en butée en mode automatique}

Action: Émission d'alarme, maintient du mode automatique.

\subsection{Surcharge actionneur (fort courant)}
Cet évènement est caractérisé par un courant du moteur important mais inférieur
au courant de moteur bloqué.

conséquence: consommation de courant importante, perte d'efficacité du pilote, risque de perte de contrôle du navire par le pilote.
Action : émission d'un information au navigateur, réduction des mouvement de l'actionneur.

\subsection{Mouvement de l'actionneur peu efficace}
Si le navire n'a pas assez de vitesse\dots

\subsection{Crash logiciel (plantage)}

Action: mettre un chien de garde, tentative de relancer le pilote,
si le pilote était en mode automatique, tentative de reprendre ce mode,
émission d'une alarme même si le pilote reprend son fonctionnement précédant.

Le chien de garde et un dispositif déclenche une action si un système s'arête.

\subsection{Perturbation du compas}

Action: compensation par courbe de déviation, comparaison avec d'autres
informations pour détecter cet évènement.

\subsection{Défauts des valeurs des capteur internes}
Rappel: capteurs 3 axes chacun : champs magnétique, accélération(pesanteur), gyromètre.
dérive dans le temps, décalages, perturbations électromagnétiques\dots

Conséquence: mauvaise performance du pilote, perte de contre du navire.
Action: Mettre des valeurs limites, des valeurs d'étalonnage et des fonctions de contrôle et de diagnostic dans les algorithmes, en cas de défaut sérieux : émission d'une alarme.

\subsection{Données reçues, erronées}

Action: introduction d'un contrôle des données dans l'algorithme.

\subsection{Données échangées : flot élevé ou mauvais format}

Conséquence: risque de perturbation du pilote ou de crash.

Action: Programmation défensive, rejet de données mal formatées, dimensionnement
pour accepter un débit maximal.

\subsection{Orientation du navire hors limites}
Gîte importante, assiette anormale, embardées,\dots

Action: émission d'information ou d'alarme

\subsection{actes de malveillance, attaque informatique}



\chapter{Principes des algorithmes}

\section{Actions du pilote}
Les actions du pilote sont les suivantes:
\begin{itemize}
	\item Commande de la barre;
	\item Affichage par voyants ou sur écran;
	\item Déclenchement de son alarme sonore;
	\item Transmission de données.
\end{itemize}
\section{Données}

\begin{itemize}
	\item Données d'état : mode, cap à suivre\dots;
	\item Orientation du navire, cap et autres information des capteurs MEMS;
	\item Vitesse surface et fond, cap et direction;
	\item Historique récent des mouvement du navire;
	\item Historique récent des commandes de l'actionneur;
	\item Caractéristiques du navire, nom, taille, type,\dots;
	\item Valeurs d'étalonnage des capteurs MEMS;
	\item Valeurs caractéristiques du navire: valeurs saisies,
     	déterminées par étalonnage et déterminées par apprentissage; 
\end{itemize}


\section{Principes de l'algorithme}

Sans capteur d'angle de barre, évaluation de la position de la barre avec le courant du moteur la tension et la durée des commandes.

Adaptation du gain du pilote par apprentissage et stockage de la valeur
dans la mémoire du pilote.

Choix de l'amplitude des commande de l'actionneur avec les compromis de consommation
de maintient de cap de confort.

A chaque mouvement de la barre évaluer son effet sur le cap.
Dans le cas d'une gîte moyenne significative, si la gîte augmente, le cap suit
la gîte et le navire cavale (augmente sa vitesse) laisser gagner au vent.

Si les mouvements du navire et de l'actionneur sont important réduire avec
l'accord du navigateur (il ne faut pas empaner) les mouvements.
enregistrer les évolutions récentes de l'orientation du navire et tenter de
prédire les prochains mouvements pour adapter les prochaines commandes de l'actionneur.
Ce fonctionnement est intéressant avec une houle régulière.

Prendre les dispositions pour réduire la propagation des erreurs d'arrondi
et empêcher les calculs impossibles (division par zéro,\dots).

En mode automatique après une situation d'alarme maintenir le mode automatique
tant que possible.


\end{document}