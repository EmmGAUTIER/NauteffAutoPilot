\documentclass[a4paper,11pt]{report}
\usepackage[utf8]{inputenc}
\usepackage[T1]{fontenc}
\usepackage[utf8]{inputenc}
\usepackage{lmodern}
%%\usepackage[francais]{babel}
\usepackage[french]{babel}
\usepackage{numprint}
\usepackage{multirow}
\usepackage{glossaries}
\title{Nauteff P-1 Spécification}
\author{Emmanuel Gautier}


%%\newglossaryentry{MEMS}{name=Nom, description={}}

\newglossaryentry{NMEA}
{
    name={NMEA}, % le terme à référencer (l'entrée qui apparaitra dans le glossaire)
    description={National Marine Electronics Association}, % la description du terme(sans retour à la ligne)
    sort={NMEA}, % si le mot contient des caractère spéciaux, ils ne seront pas pris en compte
    plural={NMEA} % la forme plurielle du terme
    }

\newglossaryentry{I2C}{name=I2C, description={Inter-Integrated Circuit en englais. C'est un bus de communication série à 2 fils à courte distance}}

\newglossaryentry{MEMS} {
    name=MEMS,
    description={Micro Electro Mechanicals Systems}
    }

\newglossaryentry{UART} {
    name=UART,
    description={Universal Asynchronous Receiver Transmitter soit émetteur-récepteur asynchrone universel, aussi appelée liaison série.}
    }

\newglossaryentry{pilonnement} {
    name=Pilonnement,
    sort={PILONNEMENT},
    description={Mouvement vertical du navire, c.à.d. de bas en haut et de haut en bas}
    }

\newglossaryentry{cavalement} {
    name=Cavalement,
    sort={CAVALEMENT},
    description={Mouvement le long de l'axe longitudinal du navire, c.à.d. d'avant en arrière, ce mouvement correspond à une accélération ou un ralentissement}
    }

\newglossaryentry{embardee} {
    name=embardée,
    sort={EMBARDEE},
    description={Mouvement latéral du navire, c.à.d. vers babord ou tribord, souvent liée à la dérive}
    }

\newglossaryentry{babord} {
    name=bâbord,
    sort={BABORD},
    description={désigne le côté gauche d'un navire, en tournant le dos à la poupe}
    }

\newglossaryentry{tribord} {
	name=tribord,
	description={le côté droit d'un navire, en tournant le dos à la poupe}
    }

\newglossaryentry{lacet} {
	name=lacet,
	description={mouvement de rotation par rapport à l'axe vertical du navire}
    }

\newglossaryentry{roulis} {
	name=roulis,
	description={ Mouvement d'oscillation d'un bateau autour de l'axe longitudinal (en général sous l'effet de la houle ou du vent)}
    }

\newglossaryentry{tangage} {
	name=Tangage,
	description={mouvement de rotation par rapport à l'axe transversal du navire}
    }

\newglossaryentry{gite} {
	name=gîte,
	description={inclinaison latérale d'un navire}
    }


\makeglossaries
\makeglossary

\begin{document}
\maketitle

\begin{abstract}
Ce document contient la spécification du prototype Nauteff P-1.
\end{abstract}

\tableofcontents
%\printglossary

\chapter{Buts du document}
\section{Buts}
Ce document décrit la spécification du Nauteff P1. Il sert de référence
pour tous les acteurs du projet.
Il contient les exigences techniques du Nauteff,
la description de son comportement et de ses fonctions. 

\section{Guide de lecture}
C'est le document de référence principal décrivant
les exigences techniques et opérationnelles et la description
des fonctions du système. Il est destiné à tous les acteurs
participant à la définition du système ou à sa réalisation.

Bien que décrivant le système il n'est pas un manuel d'utilisation.

Le détail des fonctions de calculs et algorithmes est contenu
dans un autre document à venir.

\chapter{Documents applicables et de référence}

NMEA revealed : https://gpsd.gitlab.io/gpsd/NMEA.html
Ce document sert de référence pour le développement du Nauteff~P-1.

\printglossary[numberedsection, title=Terminologie]

\chapter{Présentation du système}
\section{Objectifs du système}
Nauteff P-1 est une maquette de pilote automatique pour navires à voile ou à moteur d'une taille comprise entre 5m et 24m.
Cette maquette sert à évaluer et mettre au point le pilote automatique et améliorer son logiciel.

L'objectif d'un Pilote automatique sur un bateau 
est d'agir sur sa barre afin de maintenir le cap d'un navire.
Il doit aussi traiter des informations sur les conditions
de navigation et le fournir à d'autres appareils.
Il utilise les informations de ses capteurs, et lorsqu'elles
sont disponibles celles d'autres appareils.

\section{Environnement du système}
L'environnement du Nauteff~P-1 comprend:
\begin{itemize}
	\item un ordinateur avec les outils de développement et la sonde;
	\item un alimentation 12V;
	\item un vérin électrique;
	\item des instruments de navigation: GPS, loch, gyrouette,\dots
	\item un ou plusieurs équipiers du navire.
\end{itemize}

Nauteff P-1 est principalement utilisé et stocké dans un bureau
au chaud et au sec. Lors des essais il est à l'intérieur d'un navire
et modérément protégé de l'humidité.

Il est relié à un ordinateur par l'intermédiaire d'un sonde qui
alimente la calculateur et qui permet de charger le logiciel
de contrôler l'exécution et de faire le débogage.

Au bureau l'ordinateur est de type station de travail et l'alimentation
de l'actionneur est fourni par une alimentation stabilisée.
Lors des essais à bord d'un navire le l'ordinateur est un petit modèle
portable, et économe en courant; l'ensemble est alors alimenté par
la batterie de servitudes du navire.

L'équipage confie la commande de la barre au Nauteff pour réaliser d'autres
tâches ou pour se reposer. Il attend un bon suivi de cap, exige d'être
alerté si le Nauteff ne peut plus remplir sa fonction.

\subsection{Interfaces du système}

\subsubsection{Alimentation électrique}
Nauteff fonctionne avec une alimentation électrique
très basse tension nominale de 12V et l'alimentation de la sonde.
l'alimentation 12V répond aux exigences de la section exigences de l'environnement.


\subsubsection{Clavier}
Nauteff comporte un clavier à 6 touches fugitives (aussi appelées à contacts NO).

\subsubsection{Bus de communication UART, \ldots}
Nauteff comporte une interface série RS-232 principalement utilisée
pour la communication avec un ordinateur pour le développement et
la mise au point.
La norme \gls{NMEA} prévoit un bus RS-422, cependant cette interface
permet des échanges d'information avec des appareils utilisant
ce protocole et peu exigeants.

\subsubsection{Capteurs MEMS}
Nauteff P-1 comporte les capteurs \gls{MEMS} suivants :
\begin{enumerate}
\item un magnétomètre (LSM6DS33);
\item un gyromètre (LPS25H);
\item un accéléromètre(LPS25H);
\item un capteur de pression atmosphérique (LIS3MDL);
\item des capteurs de température.
\end{enumerate}
Ces capteurs sont dans les circuits intégrés de ST Microelectronics et montés
sur une carte d'évaluation Pololu AltIMU-10 v5. La communication entre ces capteurs et le STM32 utilise un bus \gls{I2C}. Les capteurs de température et de pression atmosphérique nz sonrt pas utilisés.

\subsubsection{Commande du moteur}
Nauteff comporte une unité de commande et de contrôle du moteur de l'actionneur.
Cette unité permet de commander la marche du moteur de l'actionneur
dans les deux sens et de mesurer l'effort du moteur (couple),
de détecter la butée, un courant excessif ou un court-circuit et un courant nul
ou anormalement faible et de surveiller la tension d'alimentation.
La commande du moteur comprend aussi une estimation de la position du moteur.

\subsubsection{Alarme sonore}
L'alarme sonore appelle l'attention du navigateur que le Nauteff ne peut plus
assurer ses fonctions ou que des valeurs sont en dehors des limites fixées
et que la navigateur doit intervenir.

\subsubsection{Écran}
Dans la version d'évaluation Nauteff P-1 n'a pas d'écran et sort les information
sur USART2 pour un affichage sur l'écran de l'ordinateur de développement.
Ça serait dommage de se priver d'un écran, il en faudra au moins un petit avec quelques caractères.

\subsubsection{Voyants}
Nauteff P-1 comporte les voyants suivant :
\begin{itemize}
	\item Voyants de mise sous tension;
	\item Voyant vert indiquant l'enclenchement du pilote;
	\item Voyant rouge lors d'un mouvement vers tribord de l'actionneur;
	\item Voyant vert lors d'un mouvement vers tribord de l'actionneur;
	\item Voyant indiquant le fonctionnement de l'USART;
	\item Selon les besoins de mise au point.
\end{itemize}
Rappel : Le rouge indique le côté bâbord (gauche) et le vert indique le côté tribord.

\subsection{Utilisateurs}
Les utilisateurs sont les concepteurs du système ou des navigateurs prenant part à son développement.
Le prototype n'est pas destiné aux utilisateurs.

\subsection{Environnement en contexte opérationnel}

\subsection{Environnement de développement et de mise au point}

\chapter{Spécification générales}
\section{Description des services attendus}

\section{Description des générale des fonctions}
\section{Exigences opérationnelles}
\subsection{Contraintes d'exploitation}
\subsection{Modes de fonctionnement}
Nauteff comporte un mode veille, un mode Automatique et un mode paramétrage.
\subsection{Transition entre modes}
\subsection{Exigences de l'environnement}
Nauteff P-1 est conçu pour supporter une ambiance saline avec une humidité
saturante permanente et un température de fonctionnement comprise entre   0\degres C et +30\degres C.
Nauteff doit aussi supporter des vibrations.
La tension d'alimentation du Nauteff doit être comprise entre 10V et 20V, il consomme moins de 50~mA en fonctionnement, hors moteur.
Il permet de commander un actionneur avec une charge inductive consommant jusqu'à
12\ A en régime permanent.
Nauteff est protégé contre les surcharges et court-circuits de l'actionneur et 
contre les inversions de polarité de l'alimentation.
\subsection{Capacité}
\subsection{Performances}
\begin{tabular}{|c|c|}
	\hline 
	Temps de démarrage &  < 2s \\ 
	\hline 
	Entrée NMEA& décodage de trames transmises à \numprint{230400} bps \\ 
	\hline 
	Réaction à une embardée & 1s max \\ 
	\hline 
	&  \\ 
	\hline 
	&  \\ 
	\hline 
\end{tabular}


\subsection{Paramétrage}
Le Nauteff P-1 comporte des paramètres suivants:
\begin{itemize}
	\item caractéristiques du moteur: courants en butée et à vide, inertie;
	\item valeurs d'étalonnage des capteurs;
	\item vitesse de réaction;
	\item amplitude de réaction;
	\item compensation d'\gls{embardee};
	\item angle de virement de bord et amplitude du mouvement de l'actionneur;
	\item valeurs limites;
	\item coefficients de pilote à définir;
	\item \ldots
\end{itemize}
Dans la version d'évaluation ces paramètres sont dans des constantes
("codées en dur") modifiées avec les outils de développement.

\subsection{Contraintes entre le matériel et le logiciel}
\subsection{Sûreté de fonctionnement}

\subsection{Exigence organisationnelles}

\subsection{Exigences de mise au point et de maintenance}
Nauteff P-1 est un prototype monté sur carte d'essai et permet un
accès facile à tous les signaux et des modifications rapides et aisées.

\section{Exigences techniques}



\section{Interface homme machine}
Nauteff est équipé d'un clavier à 6 touches fugitives (contact NO).
\\
\begin{tabular}{|c|c|c|c|}
	\hline 
	Libellé    & couleur & fonction principale &  \\ 
	\hline 
	``Auto''   & noire & Mise en mode automatique &  \\ 
	\hline 
	``Veille'' & noire & Mise en veille &  \\ 
	\hline 
	``+1''     & verte & Changement de cap 1\degres~vers tribord &  \\ 
	\hline 
	``-1''     & rouge & Changement de cap 1\degres~vers bâbord  &  \\ 
	\hline 
	``+10''    & verte & Changement de cap 1\degres~vers tribord &  \\ 
	\hline 
	``-10''    & rouge & Changement de cap 1\degres~vers bâbord  &  \\ 
	\hline 
\end{tabular} 



\chapter{Description des services}

\section{Lecture du clavier}

\begin{tabular}{|c|p{6cm} |}
	\hline
	 Touche & Action  \\ \hline
	\multicolumn{2}{|c|}{En mode veille} \\ \hline
	 Appui sur ``Auto'' & Passage en mode suivi de cap\\ \hline
	 Appui sur Veille & Sans effet \\ \ hline
	 Appui +1 &  Mise en marche de l'actionneur vers tribord \\ \hline
	 Appui -1 &  Mise en marche de l'actionneur vers bâbord \\ \hline
	 +10 & Sans effet     \\ \hline
	 -10 &  Sans effet    \\ \hline
	\multicolumn{2}{|c|}{En mode auto} \\ \hline
	pression sur touche ``Auto'' & changement de mode : suivi de cap, suivi de vent et suivi GPS     \\ \hline
	Pression sur ``Veille'' & Mise en veille     \\ \hline
	+1 & incrément de 1\degres de la consigne      \\ \hline
	 &      \\ \hline
	 
\end{tabular} 

L'appui sur la touche ``Auto'' met le Nauteff en mode automatique de suivi de cap.
La touche ``Veille''met le Nauteff en mode veille.
En mode veille l'enfoncement des touches ``+1'' et ``-1'' commandent le déplacement du vérin vers tribord et bâbord.
En mode automatique, un appui sur les touches ``+1'' et ``-1'' ``+10'' et ``-10''
changent le cap de 1\degres~ ou 10\degres~ vers tribord ou bâbord.
Le maintient enfoncé des touches ``+1'' et ``-1''

\subsection{Veille}


\subsection{Maintient du cap}

\subsection{Configuration et réglages}

\section{Détermination de l'orientation du navire}

\section{Maintien du cap}
\gls{cavalement}
\chapter{Description des informations}
\gls{embardee}

\end{document}
