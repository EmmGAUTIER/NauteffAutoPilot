\documentclass[a4paper,11pt]{report}
\usepackage[utf8]{inputenc}
\usepackage[T1]{fontenc}
\usepackage[utf8]{inputenc}
\usepackage{lmodern}
%%\usepackage[francais]{babel}
\usepackage[french]{babel}
\usepackage[xindy]{glossaries}
\makeglossaries

\title{Nauteff Notes de conception}
\author{Emmanuel Gautier}

\begin{document}
\maketitle

\begin{abstract}
Ce document contient des notes de conception du prototype Nauteff.
\end{abstract}

\tableofcontents
\chapter{Buts du document}
Ce document, dans sa version préliminaire contient des notes
sur le développement du prototype du Nauteff.
Il est destiné au développeur et aux personnes impliquées dans son
développement.

\chapter{Documents applicables et de référence}
Rédaction réservée.
\chapter{Terminologie}
\printglossary

\newglossaryentry{MEMS}
{
  name={MEMS},
  description={Micro Electromechanical Systems, microsystème électromécanique en français.}
}

\chapter{Environnement du Nauteff}
\chapter{Matériel}
\section{STM32}
\subsection{Horlogerie}
Nauteff utilise l'horloge interne HSI à 8Mhz.

\subsection{Interruptions}
\subsubsection{Liste}
\begin{tabular}{|c|c|p{7cm}|}
\hline
Interruption   & \no & Usage\\
\hline
\multicolumn{3}{|c|}{Exceptions du système}  \\
\hline
Reset               &  & appel de la fct main() \\
NMI                 &  &                      \\
HardFault           &  &                      \\
MemManage           &  &                      \\
BusFault            &  &                      \\
UsageFault          &  &                      \\
SVCall              &  & vPortSVCHandler  (fct de FreeRTOS)     \\
PendSV              &  & xPortPendSVHandler (fct de FreeRTOS)  \\
SysTick             &  & xPortSysTickHandler  (fct de FreeRTOS )\\
\hline
\multicolumn{3}{|c|}{Événements extérieurs}   \\
\hline
EXTI0          &  & Bouton Marche          \\
EXTI1          &  & Bouton Veille          \\
EXTI2          &  & Touche +1              \\
EXTI3          &  & Touche -1              \\
EXTI4          &  & Touche +10             \\
EXTI10\_15     &  & Touche -10             \\
I2C1\_Event    &  & Évènement I2C1         \\
I2C1\_Error    &  & Erreur I2C1            \\
USART1\_Event  &  & Événement USART 1      \\
\hline 
\end{tabular} 
\subsubsection{USART}
\subsubsection{I2C1}

\subsection{Affectation des broches du STM32}
Le STM32F303K8 comporte 32 broches.
\\
   \begin{tabular}{|l|l|c|c|l|}
   \hline
   STM & Nucleo& Fonction & Périphérique & Usage\\
   \hline
   PA0  & A0  & Entrée GPIO, PU   & EXTI0       & Bouton marche          \\
   PA1  & A1  & Entrée GPIO, PU   & EXTI1       & Bouton veille          \\
   PA2  & A7  & Entrée GPIO, PU   & EXTI2       & Touche +1              \\
   PA3  & A2  & Entrée GPIO, PU   & EXTI3       & Touche -1              \\
   PA4  & A3  & Entrée GPIO, PU   & EXTI4       & Touche +10             \\
   PA5  & A4  & Entrée GPIO       & EXTI5       & Inutilisable sur nucléo\\

   PA6  & A5  & Entrée GPIO       & EXTI6       & Inutilisable sur nucléo\\
   PA7  & A6  & Entrée analogique & ADC2\_IN4   & Courant moteur         \\
   PA8  & D9  & Alt Fct \no 0     & RCC\_MCO    & Sortie horloge système \\
   PA9  & D1  & Alt Fct \no 7     & USART1 TX   & Sortie série           \\
   PA10 & D0  & Alt Fct \no 7     & USART1 RX   & Entrée série           \\
   PA11 & D10 & Entrée GPIO PU    & EXTI10\_15  & Touche -10             \\
   PB0  & D3  & Sortie GPIO       & GPIO B0     & Commande moteur        \\
   PB1  & D6  & Entrée analogique & ADC1\_IN2   & Tension alim.          \\
   PB3  & D13 & Sortie GPIO       & GPIO B3     & LED Verte et embrayage \\
   PB4  & D12 & Sortie GPIO       & GPIO B4     & Sens pont A (INA)      \\
   PB5  & D11 & Sortie GPIO       & GPIO B5     & Sens pont B (INB)      \\
   PB6  & D5  & Alt Fct \no 4     & I2C1 SCL    & Vers capteurs MEMs     \\
   PB7  & D4  & Alt Fct \no 4     & I2C1 SDA    & Vers capteurs MEMs     \\
   \hline
\end{tabular}
   La broche PA8 sert à observer l'horloge système à l'oscilloscope. Elle pourra servir plus tard à une autre fonction.

\subsection{Organisation de la mémoireSTM32}

\begin{tabular}{|c|c|c|c|p{5cm}|}
	\hline 
	Mémoire & Adresse     & taille &  &  \\ 
	\hline 
	FLASH   & 0x0800 0000 & 64K &  &  Permanente, plus lente \\ 
	RAM     & 0x2000 0000 & 12K &  &  \\ 
	CCM RAM & 0x1000 0000 &  4K &  &  Core coupled memory\\ 
	\hline
\end{tabular}

\chapter{Contexte de la conception}
\section{présentation de l'application}
Nauteff est un pilote automatique pour navires.
\section{Principales exigences applicables}
\subsection{Exigences liées à la sûreté de fonctionnement}
Le navigateur confiant la timonerie de son navire à Nauteff,
ce dernier doit être programmé pour éviter toute panne ou
comportement mettant en danger le navire et son équipage.
La programmation suit le principe KISS ("Keep It Simple Stupid")
et est défensive.
Les principes suivants sont adoptés:
\begin{itemize}
	\item Absence d'allocation dynamique après le démarrage ;
	\item Usage très réduit de variables locales dans les fonctions.
\end{itemize}


\subsection{Architecture matérielle opérationnelle}
Nauteff utilise une carte Nucleo dotée d'un STM32~F303K8
et d'une sonde ST-Link accessible par prise USB.

Le STM32~F303K8 ne comporte que 32 broches, le prochain Nauteff
devrait utiliser un STM32 avec au moins 48 broches.
\subsection{Logiciels imposés}
Néant.
\subsection{Interface avec d'autres applications}
Interfaces NMEA 0183
Cette interface est aussi utilisée pour la sortie de messages
pendant le développement.
\section{Contraintes de développement}
\subsection{Méthode et formalisme}
La programmation est essentiellement fonctionnelle.
Le formalisme est celui d'UML pour les schémas.
Les concepts de la programmation objets ne sont pas utilisés.

\subsection{Langage de programmation}
La programmation est réalisée entièrement avec le langage C.
Seules quelques fonctions de la librairie standard et de la librairie
mathématiques sont utilisées. Le langage C permet d'accéder très facilement
à la mémoire et surtout aux registres des périphériques.
Le recours à l'assembleur n'est pas envisagé.
\subsection{Logiciels extérieurs}
FreeRTOS V 10.0.1 fournit le noyau temps réel, les horloges,
les sémaphores et les canaux de communication.
Nauteff P-1 utilise la librairie standard libc et la librairie mathématiques lm. 
\subsection{Outils}

\begin{tabular}{|c|c|c|}
	\hline 
	EDI& Eclipse C/C++ & eclipse.org    \\ 
	\hline 
	Compilation & arm-none-eabi-gcc & chaîne du GNU, compilation croisée \\ 
	\hline 
	sonde & ST-Link & intégrée sur carte Carte Nucleo   \\ 
	\hline 
	Accès sonde & openocd 0.9.0 &   \\ 
	\hline 
	Documentation & Latex &  \\ 
	\hline 
\end{tabular}

\subsection{Environnement matériel de développement}
Le développement au bureau est réalisé avec les matériels suivant :
\begin{itemize}
	\item un PC permettant de faire fonctionner eclipse;
	\item un oscilloscope pour certaines mise au point ;
	\item occasionnellement un voltmètre;
	\item petit outillage: pinces tournevis, dénudeuse,\ldots;
\end{itemize}
Lors des essais en mer du prototype un petit PC consommant peu
mais ne permettant pas d'utiliser eclipse permet de faire des
modifications du logiciel et, en particulier, des ajustements
de paramètres.Il embarque la chaîne de compilation et openocd.
La compilation est réalisée avec la commande make.
\chapter{Architecture du logiciel}
\section{Noyau temps réel : FreeRTOS}
\section{Tâches}
\begin{tabular}{|l|c|l|}
\hline 
Tâche & Priorité & Nom \\
\hline 
Gestion du clavier & 6 & \texttt{TaskKeyboard}\\
Contrôle du moteur & 5 & \texttt{TaskMotor}\\
Gestion de l'état & 3 & \texttt{TaskCore}\\
Gestion des capteurs MEMs & 3 & \texttt{TaskMEMs}\\
\hline 
\end{tabular} 
\subsection{ScrutationClavier}
La tâche utilise la fonction nommée TaskKeyboard qui n'utilise aucun argument.
Elle fonctionne par scrutation périodique des entrées sur lesquelles sont branchées les touches.
La scrutation périodique est utilisée car les touches provoquaient de très nombreux
rebonds parasites. 
Elle envoie les ordres à la tâche de gestion de l'état
par l'intermédiaire d'un \texttt{MessageBuffer\_t} nommé buffclav.

\subsection{Contrôle du moteur}
Cette tâche reçoit les ordres de la tâche de gestion,
elle commande l'embrayage du moteur et son fonctionnement : sens et durée.
Elle corrige la durée de fonctionnement en fonction de l'effort du moteur.
Lorsque le moteur tourne, elle mesure le courant du moteur,
vérifie que le moteur n'est pas bloqué ou débranché et
renvoie une estimation de l'effort du moteur. Elle informe la tâche principale
lorsque le moteur a fini un mouvement demandé.
\subsubsection{Entrées}

\begin{itemize}
  \item Ordres de commande du moteur et de l'embrayage en provenance de la tâche principale;
  \item valeurs du courant et de la tension en provenance de l'ADC;
  \item horloge (à définir).
\end{itemize}
\subsubsection{Sorties}
\begin{itemize}
	\item ordres vers le moteur mise en marche et sens;
	\item ordres vers l'embrayage : enclenchement et désenclenchement ;
	\item informations vers la tâche principale sur le fonctionnement du moteur estimation de l'effort, de la position, blocage ou court-circuit.
\end{itemize}
\subsection{Gestion des capteurs MEMS}
Cette tâche lit les informations des capteurs MEMs, les filtre,
calcule le cap et le lacet et envoie les informations traitées à la tâche principale.
Le signal d'interruption n'étant pas disponible sur la carte de Pololu
la tâche fonctionne par scrutation.
\subsubsection{Sorties}
La tâche envoie les données vers la tâche principale.
\subsection{Tâche principale}
\subsubsection{Entrées}
\begin{itemize}
  \item Commandes en provenance du clavier;
  \item Informations du moteur;
  \item Informations des capteurs MEMS ;
  \item Informations en provenance des interfaces NMEA
  \item horloge (à définir).
\end{itemize}
\subsubsection{Sorties}
\begin{itemize}
  \item Commandes moteurs;
  \item alarmes;
\end{itemize}

\section{Canaux de communications}

\begin{tabular}{|l|c|c|c|}
\hline
Canal & Nom & Type & Taille \\
\hline
Messages vers le gestionnaire principal & \texttt{bufferCore} & \texttt{MessageBuffer\_t} & 10 \\
\hline 
Ordres vers le moteur & \texttt{bufferMotor} & \texttt{MessageBuffer\_t} & 5 \\ 
\hline
\end{tabular}
\subsection{Message gestionnaire principal}
Ces messages sont envoyés par les tâches TaskKeyboard, TaskMotor et TaskMEMs.
Il sont stockés dans une structure dont les deux premiers octets comportent
un code indiquant la nature du message et, en option, d'autres informations.
\\
\begin{tabular}{|l|l|}
\hline 
Ordre & Code et valeurs optionnelles \\
\hline
\multicolumn{2}{|c|}{Ordres du clavier} \\
\hline 
Mise en mode automatique & MSG\_KBD\_AUTO \\ 
Mise en veille & MSG\_KBD\_STDBY \\ 
+1 & MSG\_KBD\_STARBOARD\_ONE \\ 
-1 & MSG\_KBD\_PORT\_ONE \\ 
+10 & MSG\_KBD\_STARBOARD\_TEN \\ 
-10 & MSG\_KBD\_PORT\_TEN \\ 
+1 relachée & MSG\_KBD\_STARBOARD\_ONE\_END \\ 
-1 relachée & MSG\_KBD\_PORT\_ONE\_END \\ 
\hline 
\multicolumn{2}{|c|}{Informations en provenance de la tâche de gestion du moteur} \\
\hline
Position estimée du moteur & MSG\_MOT\_POSITION\_ESTIMEE \\
Effort moteur & MSG\_MOT\_EFFORT + float effort\\ 
Moteur bloqué ou en butée & MSG\_MOT\_BLOCKED \\
Court-circuit sortie moteur & MSG\_MOT\_SHORT\_CIRCUIT \\
Fin de déplacement actuateur & MSG\_MOT\_END\_MOVE + float : effort \\
\hline
\multicolumn{2}{|c|}{Informations des capteurs MEMs}\\
\hline
Info Capteurs MEMs & MSG\_MEMS\_VALUES + 2 floats : cap et lacet \\
Info \gls{MEMS}MEMs indisponible & MSG\_MEMS\_FAILURE \\

\hline
\end{tabular} 
\subsection{Commandes envoyées à la tâche de gestion du moteur}
Ces messages sont envoyés par la tâche principale ou un timer
à la tâche de gestion du moteur.
\\
\begin{tabular}{|l|l|}
	\hline 
	Ordre & Code et valeurs optionnelles \\ 
	\hline
	\multicolumn{2}{|c|}{Commande de la tâche principale} \\
	\hline 
    Enclenchement de l'embrayage & MSG\_MOT\_EMBRAYE\\
    Désenclenchement de l'embrayage & MSG\_MOT\_DEBRAYE\\
    Mise en marche vers tribord & MSG\_MOT\_STARBOARD\\
    Mise en marche vers babord  & MSG\_MOT\_PORT\\
    Mouvement vers tribord & MSG\_MOT\_MOVE\_STARBOARD + float : valeur\\
    Mouvement vers babord  & MSG\_MOT\_MOVE\_PORT + float: valeur\\
    Arrêt du moteur & MSG\_MOT\_STOP\\
    Demande de mesure d'effort & MSG\_MOT\_DMD\_EFFORT\\
    \hline
	\multicolumn{2}{|c|}{commande du timer} \\
	\hline
	Commande du TIMER & MSG\_MOT\_MOVING\_CONTROL\\
    \hline
\end{tabular} 

La tâche de gestion du moteur effectue les déplacements
dont l'amplitude doit être proportionnelle à la valeur en paramètre,
elle adapte la durée au courant du moteur.


\end{document}
